% -*-latex-*-
\begin{nusmvCommand} {read\_trace} {Loads a previously saved trace}

\cmdLine{read\_trace [-h | [-i filename] [-u] [-s] filename]}

\begin{cmdOpt}

\opt{-i \parameter{\filename{filename}}}{ Reads in a trace from the
  specified file. Note that the file must only contain one trace.
  \emph{This option has been deprecated}. Use the explicit filename
  argument instead.  }

\opt{-u}{ Turns ``undefined symbol'' error into a warning. The loader
  will ignore assignments to undefined symbols.}

\opt{-s}{ Turns ``wrong section'' error into a warning. The loader
  will accept symbol assignment even if they are in a
  different section than expected. Assignments will be silently moved
  to appropriate section, i.e. misplaced assignments to state symbols
  will be moved back to previous state section and assignments to
  input/combinatorial symbols will be moved forward to successive
  input/combinatorial section. Such a way if a variable in a
  model was input and became state or vice versa the existing
  traces still can be read and executed.}
\end{cmdOpt}

Loads a trace which has been previously output to a file with the XML
Format Output plugin. The model from which the trace was originally
generated must be loaded and built using the command ``\command{go}''
first.\\Please note that this command is only available on systems
that have the Expat XML parser library installed.
\end{nusmvCommand}
