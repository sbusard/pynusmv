% -*-latex-*-
\begin{nusmvCommand}{print\_fsm\_stats} {Prints out information about the fsm and clustering.}

\cmdLine{print\_fsm\_stats [-h] | [-m]  | [-p] | [-o output-file]}

   This command prints out information regarding the fsm and each cluster.
   In particular for each cluster it prints out the cluster number, the size
   of the cluster (in BDD nodes), the variables occurring in it, the size of
   the cube that has to be quantified out relative to the cluster and the
   variables to be quantified out. 

   Also the command can print all the normalized predicates the FMS
   consists of. A normalized predicate is a boolean expression which
   does not have other boolean sub-expressions. For example,
   expression \code{(b<0 ? a/b : 0) = c} is normalized into \code{(b<0
   ? a/b=c : 0=c)} which has 3 normalized predicates inside:
   \code{b<0}, \code{a/b=c}, \code{0=c}.
   

\begin{cmdOpt}
\opt{-h }{Prints the command usage.}
\opt{-m}{Pipes the output generated by the command to the program
specified by the \shellvar{PAGER} shell variable if defined, else
through the \unix command \shellcommand{more}.}
\opt{-p}{Prints out the normalized predicates the FSM consists of.
Expressions in properties are ignored.}
\opt{-o \parameter{\filename{output-file}}}{Writes the output
generated by the command to the file \filename{output-file}.}
\end{cmdOpt}

\end{nusmvCommand}
