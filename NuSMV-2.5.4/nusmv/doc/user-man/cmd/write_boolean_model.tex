% -*-latex-*-
\begin{nusmvCommand} {write\_boolean\_model} {Writes a flat and boolean model to a file}

\cmdLine{write\_boolean\_model [-h] [-o filename]}

Writes the currently loaded SMV model in the specified file, after
having flattened and booleanized it. Processes are eliminated and a
corresponding equivalent model is printed out.

If no file is specified, the file specified via the environment
variable \envvar{output\_boolean\_model\_file} is used if any, otherwise
standard output is used.

\begin{cmdOpt}
\opt{-o \parameter{\filename{filename}}} {Attempts to write the flat
and boolean SMV model in \filename{filename}}
\end{cmdOpt}

In \NuSMV scalar variables are dumped as \reserved{DEFINEs} whose body
is their boolean encoding.
  
This allows the user to still express and see parts of the generated
boolean model in terms of the original model's scalar variables names
and values, and still keeping the generated model purely boolean.

Also, symbolic constants are dumped within a \reserved{CONSTANTS}
statement to declare the values of the original scalar variables' for
future reading of the generated file.

When \nusmv detects that there were triggered one or more dynamic
reorderings in the BDD engine, the command
\command{write\_boolean\_model} also dumps the current variables
ordering, if the option \envvar{output\_order\_file} is set.

The dumped variables ordering will contain single bits or scalar
variables depending on the current value of the option
\envvar{write\_order\_dumps\_bits}. See command \command{write\_order}
for further information about variables ordering.

\end{nusmvCommand}
