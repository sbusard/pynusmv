% -*-latex-*-
\begin{nusmvCommand} {dynamic\_var\_ordering} {Deals with the dynamic variable ordering.}

\cmdLine{dynamic\_var\_ordering [-d] [-e <method>] [-f <method>] [-h]}

Controls the application and the modalities of (dynamic) variable
ordering. Dynamic ordering is a technique to reorder the BDD variables
to reduce the size of the existing BDDs. When no options are
specified, the current status of dynamic ordering is displayed. At
most one of the options \commandopt{e}, \commandopt{f}, and
\commandopt{d} should be specified.  Dynamic ordering may be time
consuming, but can often reduce the size of the BDDs dramatically. A
good point to invoke dynamic ordering explicitly (using the
\commandopt{f} option) is after the commands \command{build\_model},
once the transition relation has been built.  It is possible to save
the ordering found using \command{write\_order} in order to reuse it
(using \command{build\_model -i order-file}) in the future.\\

\begin{cmdOpt}
\opt{-d}{ Disable dynamic ordering from triggering automatically.}

\opt{-e \parameter{\set{<method>}{sift, random, random\_pivot,
      sift\_converge, symmetry\_sift, symmetry\_sift\_converge,
      window\{2,3,4\}, window\{2,3,4\}\_converge, group\_sift,
      group\_sift\_converge, annealing, genetic, exact, linear,
      linear\_converge}}}{ Enable dynamic ordering to trigger automatically
whenever a certain threshold on the overall BDD size is reached.
\code{<method>} must be one of the following:
 }
\tabItem{{\varvalue{\bf sift}}: Moves each variable throughout the order to
                 find an optimal position for that variable (assuming all
                 other variables are fixed).  This generally achieves
                 greater size reductions than the window method, but is slower.
}
\tabItem{{\varvalue{\bf random}}: Pairs of variables are randomly chosen, and
                 swapped in the order. The swap is performed by a series of
                 swaps of adjacent variables. The best order among those
                 obtained by the series of swaps is retained. The number of
                 pairs chosen for swapping equals the number of variables
                 in the diagram.
}
\tabItem{{\varvalue{\bf random\_pivot}}: Same as {\varvalue{\bf random}}, but the two
                 variables are chosen so that the first is above the
                 variable with the largest number of nodes, and the second
                 is below that variable.  In case there are several
                 variables tied for the maximum number of nodes, the one
                 closest to the root is used.
}
\tabItem{{\varvalue{\bf sift\_converge}}: The {\varvalue{\bf sift}} method is iterated
                 until no further improvement is obtained.
}
\tabItem{{\varvalue{\bf symmetry\_sift}}: This method is an implementation of
                 symmetric sifting. It is similar to sifting, with one
                 addition: Variables that become adjacent during sifting are
                 tested for symmetry. If they are symmetric, they are linked
                 in a group. Sifting then continues with a group being
                 moved, instead of a single variable.
}
\tabItem{{\varvalue{\bf symmetry\_sift\_converge}}: The {\varvalue{\bf symmetry\_sift}}
                 method is iterated until no further improvement is obtained.
}
\tabItem{{\varvalue{\bf window\{2,3,4\}}}: Permutes the variables within windows
                 of "n" adjacent variables, where "n" can be either 2, 3 or 4,
                 so as to minimize the overall BDD size.
}
\tabItem{{\varvalue{\bf window\{2,3,4\}\_converge}}: The {\varvalue{\bf window\{2,3,4\}}} method
                 is iterated until no further improvement is obtained.
}
\tabItem{{\varvalue{\bf group\_sift}}: This method is similar to
                 {\varvalue{\bf symmetry\_sift}}, but uses more general criteria to
                 create groups.
}
\tabItem{{\varvalue{\bf group\_sift\_converge}}: The {\varvalue{\bf group\_sift}} method is
                 iterated until no further improvement is obtained.
}
\tabItem{{\varvalue{\bf annealing}}: This method is an implementation of
                 simulated annealing for variable ordering. This method is
                 potentially very slow.
}
\tabItem{{\varvalue{\bf genetic}}: This method is an implementation of a
                 genetic algorithm for variable ordering. This method is
                 potentially very slow.
}
\tabItem{{\varvalue{\bf exact}}: This method implements a dynamic programming
                 approach to exact reordering. It only stores a BDD
                 at a time. Therefore, it is relatively efficient in
                 terms of memory. Compared to other reordering
                 strategies, it is very slow, and is not recommended
                 for more than 16 boolean variables.
}
\tabItem{{\varvalue{\bf linear}}: This method is a combination of
                 sifting and linear transformations.
}
\tabItem{{\varvalue{\bf linear\_converge}}: The {\varvalue{\bf linear}} method is
                 iterated until no further improvement is obtained.
}            

\opt{-f \parameter{\set{<method>}{sift, random, random\_pivot,
      sift\_converge, symmetry\_sift, symmetry\_sift\_converge,
      window\{2,3,4\}, window\{2,3,4\}\_converge, group\_sift,
      group\_sift\_converge, annealing, genetic, exact, linear,
      linear\_converge}}}{Force dynamic ordering to be invoked immediately.
The values for \code{<method>} are the same as in option
\commandopt{e}. }

\end{cmdOpt}

\end{nusmvCommand}
